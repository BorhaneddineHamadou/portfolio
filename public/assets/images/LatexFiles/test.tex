%\documentclass[a4paper,10pt,twocolumn]{article} %Pour présenter l'article sous forme de 2 colonnes
%Pour un document multicolonnes, il suffit d'ajouter apres begin document :
% begin{multicols}{4}
\documentclass{article} %Type de document
\usepackage[utf8]{inputenc}
\usepackage[T1]{fontenc}
\usepackage{fancybox} %to use doublebox,ovalbox & shadowbox
\usepackage{color}
%\pagecolor{red}
\usepackage[left=2cm, right=2cm, top=2cm, bottom=2cm]{geometry} %modify marging
\usepackage{hyperref} %pour mettre les liens du sommaire vers différents pages
\usepackage{graphicx} %permettre d'inserer les images
\usepackage[tikz]{bclogo}
\usepackage{verbatim} %Pour afficher les source codes
\newcommand{\be}{\begin{enumerate}} %we can now use \be a la place de \bein{enumerate}
\title{Mon document latex} %titre du document
\author{Auteur du document}
\date{07/08/2021} %Date de création
\begin{document}
\renewcommand{\contentsname}{Table des matière} %changer le nom du table de matière
\tableofcontents
\maketitle{} %Génération du titre
\section{Ma première section ici}
\subsection{Partie 1}
Je peux aussi faire des sous sections
je mets ici le contenu de ma sous section
\\je peux aussi revenu à la ligne !
\\Et écrire autant de lignes que je veux 
\\
\textit{Mon text en italic}
\\
\textbf{Voici un text en gras}
\\
\underline{Je souligne mon text}
\center{text centré}
\\
\flushright{Alignement à droite}
\flushleft{Alignement è gauche}
\newpage
Hello
\\
\hspace{2cm} \vspace{2cm} Here is a new page
\\ Hello in the new page
\vspace{2cm}
\\
\tiny{Text avec taille miniscule}
\\
\small{Text avec taille petite}
\\
\normalsize{Text avec taille normale}
\\

{\Huge Taille très grande !
\\All text here is Huge !
}
\\
\vspace{2cm}
\fbox{
\begin{minipage}{5cm} %5cm c'est le largeur du text encadré
1ère ligne du text encadré !
\\
2ème ligne du text encadré !
\\
3ème ligne du text encadré !
\end{minipage}
}
\\
\vspace{1cm}
\shadowbox{Text ombré !}
\\
\vspace{1cm}
\doublebox{Text doublement encadré}
\\
\vspace{1cm}
\ovalbox{Text dans un cadre a coins arrondis}
\\
{\center
\line(1,0){200} % x = 0 et y = 1 et 250 largeur
\line(0,1){200}
}
\\
Voici les modules que vous devez valider :
\begin{itemize}
\item Analyse
\item Proba
\item Algo
\end{itemize}
Voici les memes modules mais cette fois enumerés :
%\be
\begin{enumerate}
\item Analyse
\item Proba
\item Algo
\end{enumerate}
Une liste descriptive :
\begin{description}
\item[Analyse : ] Analyse 1, Analyse 2
\item[Algèbre : ] Algèbre 1, Algèbre 2
\item[Algo : ] Algo 1, Algo 2
\end{description}
\begin{tabular}{|c||c|c|} % c : contenu centré; l : left; r : right
\hline
colonne 1 & colonne 2 & colonne 3 
\\
\hline
1 & 2 & 3
\\
\hline
4 & 5 & 6
\\
\hline
\end{tabular}
\\
\vspace{1cm}
\textcolor{red}{\textbf This text is Red !}
\textcolor{blue}{\\ But this text is Blue, WOW !}

\colorbox{blue}{Text inside the box}
\\
Le lorem ipsum\footnote{Note en bas de page} est, en imprimerie, 
une suite de mots sans signification utilisée à titre provisoire 
pour calibrer une mise en page, le texte définitif venant remplacer le 
faux-texte dès qu'il est prêt ou que la mise en page est achevée.\marginpar{Une note en marge}
\\
\vspace{1cm}
\includegraphics[scale=0.5]{img/ESI.png}
\\
\vspace{1cm}
%La création d'une boite design : avec le package bclogo
\begin{bclogo}[couleur = green!30, arrondi = 0.1, ombre=true, epOmbre=0.45, couleurOmbre=black!85, logo=\bcplume]{Mon titre ici} %30 c'est l'opacité
Exemple d'une boite bclogo
\\ c'est magnifique !
\end{bclogo}
\vspace{1cm}
%Afficher un source code au sein du pdf : avec le package verbatim
\begin{verbatim}
$(document).on('click', '.todel', function () {
        var row = $(this).closest('tr');
        var item_id = row.attr('data-item-id');
        delete toitems[item_id];
        row.remove();
        if(toitems.hasOwnProperty(item_id)) { } else {
            localStorage.setItem('toitems', JSON.stringify(toitems));
            loadItems();
            return;
        }
    });
\end{verbatim}
\section{conclusion}
Je mets ma conclusion ici
Et finalement la fin de mon document latex
\\Merci !
\end{document}